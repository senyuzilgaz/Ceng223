\documentclass[12pt]{article}
\usepackage[utf8]{inputenc}
\usepackage{float}
\usepackage{amsmath}
\usepackage{tikz}


\usepackage[hmargin=3cm,vmargin=6.0cm]{geometry}
%\topmargin=0cm
\topmargin=-2cm
\addtolength{\textheight}{6.5cm}
\addtolength{\textwidth}{2.0cm}
%\setlength{\leftmargin}{-5cm}
\setlength{\oddsidemargin}{0.0cm}
\setlength{\evensidemargin}{0.0cm}

%misc libraries goes here



\begin{document}

\section*{Student Information } 
%Write your full name and id number between the colon and newline
%Put one empty space character after colon and before newline
Full Name : Ilgaz Şenyüz \\
Id Number : Ilgaz Şenyüz\\

% Write your answers below the section tags
\section*{Answer 1}
a)I chose Kruskal's algorithm. I organized edges according to their costs, minimum costs come first.\\\\
add e-f\\
add a-d\\
add g-h\\
add e-h\\
add d-b\\
add d-g\\
don't add h-f (since it forms cycle)\\
add c-f\\
add h-i\\
I've reach 8 edges, which is the number of vertices-1, so my algorithm finished. If I continue to look edges, every one of them would form a cycle, therefore ı would not add them.\\\\
b)\\\\
\begin{tikzpicture}
\draw 
(-3, 2) node[circle, black, draw](a){A}
(7, 1) node[circle, black, draw](b){B}
(5, 4) node[circle, black, draw](c){C}
(1, -4) node[circle, black, draw](d){D}
(2, 5) node[circle, black, draw](e){E}
(3, 2) node[circle, black, draw](f){F}
(1, 0) node[circle, black, draw](g){G}
(1, 3) node[circle, black, draw](h){H}
(3, -1) node[circle, black, draw](i){I};

\draw[-] (c) -- node[above] {3} (f);
\draw[-] (f) -- node[below] {1} (e);
\draw[-] (e) -- node[above] {2} (h);
\draw[-] (h) -- node[below] {4} (i);
\draw[-] (h) -- node[above] {2} (g);
\draw[-] (g) -- node[below] {3} (d);
\draw[-] (d) -- node[above] {3} (b);
\draw[-] (d) -- node[below] {1} (a);

\end{tikzpicture}
\\\\
c)The mininmum spanning tree is unique for the graph G, since we can only construct one minimum spanning tree with these edges. But in general, there can be more than 1 minimum spanning trees. Lets assume there is an edge "x" and if we add it tou or minimum spanning tree, it forms a cycle. But let assume again that the weight of this edge is also equal to minimum cost edge. Therefore we can form another minimum spanning tree by swaping "x" with the edge that makes cycle.
\\\\
d)So according to the question, we assume that minimum-weight edge of this graph is unique. And lets assume we are using Kruskal's algorithm to construct a minimum spanning tree. According to algorithm, we must start with the lowest cost edge, therefore the minimum-weight edge is included in the minimum spanning tree at the very first step, since it is one and only the lowest cost edge. Therefore in every MST, this edge is included.
\section*{Answer 2}
    First, I will look at the deegrees of vertices in both graphs.\\
    In the first graph, ${c,d,f}$ are the vertices with degree 2, and ${a,e}$ are the vertices with degree 3 and ${b}$ is the vertex with degree 4,\\
    So there is 3 vertices with degree 2, 2 with degree of 3, and 1 with degree of 4.\\\\
    For the second graph, ${p,r,o}$ are the vertices with degree 2, and ${m,n}$ are the vertices with degree 3 and ${q}$ is the vertex with degree 4,\\
    So there is 3 vertices with degree 2, 2 with degree of 3, and 1 with degree of 4.\\\\ So far everything is allright, their vertices and deegrees can match. Lets investigate more, Both of these graph's has 1 vertex with degree of 4, so they must be equal if they are isomorphic. \\\\
    So lets assume G, and H are isomorphic, then the vertex "b" matches to "q", since they are the only vertices with degree of 4.\\Then lets say "a" matches "m", and "e" matches "n". Everything holds since their degrees are equal and in G, "b" is connected to "a" and "e". And in H, "q" is connected to "m" and "n". \\Now "c" matches "p", since "c" is connected to "a", and "b", which are equivalent to "m" and "q" in graph H, and p are connected to them.\\And in the same way, "d" matches "r". Finally, "f" matches "o".\\
    Everything holds, G and H are isomorphic since their number of vertices, degrees of the vertices, and number of edges are same, and we even show at least one case holds.

\section*{Answer 3}
a) The number vertex is 7, the number of edge is 6, and the height of T is 3.
\\\\
b)
PostOrder: 13,19,23,58,43,24,17\\
InOrder: 13,17,19,24,23,43,58
\\
PreOrder:17,13,24,19,43,23,58
\\
c) Yes, T is a full binary tree, since obviously every node of T has either 0, or 2 nodes. Which means, a node of T is either a leaf node, or has both left and right child.
\\\\
d) No, T is not a complete binary tree. Because in a complete tree, all the levels except the last level need to be completely full( And in the last level tree should have all the nodes as left as possible). But in T, obviously level 1 and 2 are not full.
\\\\
e)No, T is not a balanced binary tree, because the height of its left subtree is 1, while the height of its right subtree is 3. In a balanced tree, the difference of the right and left subtrees of any node should be less or equal to 1. But in our tree, for root the difference is apparently 2.
\\\\
f)No it's not. Because the node "u" is in the right subtree of r, so it must be bigger than r's value : 24. But in fact, the value of node u is 23, and it is less than 24. So this is not a binary search tree.
\\\\
g)It is 11. Think a left skewed binary tree with height 5, it has 6 nodes. And now, every node in the left-skewed tree except the leaf node must have a right child, so that the tree is full. 6+5=11.
\\\\


\end{document}
