\documentclass[a4paper,12pt]{article}

\usepackage{float}


\usepackage[utf8]{inputenc}
\usepackage[dvips]{graphicx}
%\usepackage{a4wide}
\usepackage{epsfig}
\usepackage{fancybox}
\usepackage{verbatim}
\usepackage{array}
\usepackage{latexsym}
\usepackage{alltt}
\usepackage{amssymb}
\usepackage{amsmath,amsthm}
\usepackage{bm}
\usepackage{wasysym}
\documentclass[11pt]{article}
\usepackage[utf8]{inputenc}
\usepackage[dvips]{graphicx}
\usepackage{fancybox}
\usepackage{verbatim}
\usepackage{array}
\usepackage{latexsym}
\usepackage{alltt}
\usepackage{hyperref}
\usepackage{textcomp}
\usepackage{color}
\usepackage{amsmath}
\usepackage{amsfonts}
\usepackage{tikz}
\usepackage{fitch}  % to use fitch
\usepackage{float}
\usepackage[hmargin=3cm,vmargin=5.0cm]{geometry}
%\topmargin=0cm
\topmargin=-2cm
\addtolength{\textheight}{6.5cm}
\addtolength{\textwidth}{2.0cm}
%\setlength{\leftmargin}{-5cm}
\setlength{\oddsidemargin}{0.0cm}
\setlength{\evensidemargin}{0.0cm}

%\usepackage{fullpage}
%\usepackage{hyperref}
\usepackage{listings}
\usepackage{color}
\usepackage{algorithm}
\usepackage{algpseudocode}
\usepackage[hmargin=2cm,vmargin=3.0cm]{geometry}
%\topmargin=0cm
%\topmargin=-1.8cm
%\addtolength{\textheight}{6.5cm}
%\addtolength{\textwidth}{2.0cm}
%\setlength{\leftmargin}{-3cm}
%\setlength{\oddsidemargin}{0.0cm}
%\setlength{\evensidemargin}{0.0cm}

%misc libraries goes here
\usepackage{tikz}
\usepackage{tikz-qtree}
\usetikzlibrary{automata,positioning}

\usepackage{multicol}
\usepackage{enumitem}

\usepackage[most]{tcolorbox}

\usepackage[colorlinks=true,urlcolor=black,linkcolor=black]{hyperref}


\lstdefinestyle{customtex}{
    %backgroundcolor=\color{lbcolor},
    tabsize=2,
    language=TeX,
    numbers=none,
    basicstyle=\footnotesize\ttfamily,
    numberstyle=\footnotesize,
    aboveskip={0.0\baselineskip},
    belowskip={0.0\baselineskip},
    %
    columns=flexible,
    keepspaces=true,
    fontadjust=true,
    upquote=true,
    %
    breaklines=true,
    prebreak=\raisebox{0ex}[0ex][0ex]{\ensuremath{\hookleftarrow}},
    frame=single,
    showtabs=false,
    showspaces=false,
    showstringspaces=false,
    %
    %identifierstyle=\color[rgb]{0,0.2,0.8},
    identifierstyle=\color[rgb]{0,0,0.5},
    %identifierstyle=\color[rgb]{0.133,0.545,0.133},
    %keywordstyle=\color[rgb]{0.8,0,0},
    %keywordstyle=\color[rgb]{0.133,0.545,0.133},
    keywordstyle=\color[rgb]{0,0,0.5},
    %commentstyle=\color[rgb]{0.133,0.545,0.133},
    commentstyle=\color[rgb]{0.545,0.545,0.545},
    %stringstyle=\color[rgb]{0.827,0.627,0.133},
    stringstyle=\color[rgb]{0.133,0.545,0.133},
    %
    literate={â}{{\^{a}}}1 {Â}{{\^{A}}}1 {ç}{{\c{c}}}1 {Ç}{{\c{C}}}1 {ğ}{{\u{g}}}1 {Ğ}{{\u{G}}}1 {ı}{{\i}}1 {İ}{{\.{I}}}1   {ö}{{\"o}}1 {Ö}{{\"O}}1 {ş}{{\c{s}}}1 {Ş}{{\c{S}}}1 {ü}{{\"u}}1 {Ü}{{\"U}}1 {~}{$\sim$}{1}
}

\lstdefinestyle{output}{
    %backgroundcolor=\color{lbcolor},
    tabsize=2,
    numbers=none,
    basicstyle=\footnotesize\ttfamily,
    numberstyle=\footnotesize,
    aboveskip={0.0\baselineskip},
    belowskip={0.0\baselineskip},
    %
    columns=flexible,
    keepspaces=true,
    fontadjust=true,
    upquote=true,
    %
    breaklines=true,
    prebreak=\raisebox{0ex}[0ex][0ex]{\ensuremath{\hookleftarrow}},
    frame=single,
    showtabs=false,
    showspaces=false,
    showstringspaces=false,
    %
    %identifierstyle=\color[rgb]{0.44,0.12,0.1},
    identifierstyle=\color[rgb]{0,0,0},
    keywordstyle=\color[rgb]{0,0,0},
    commentstyle=\color[rgb]{0,0,0},
    stringstyle=\color[rgb]{0,0,0},
    %
    literate={â}{{\^{a}}}1 {Â}{{\^{A}}}1 {ç}{{\c{c}}}1 {Ç}{{\c{C}}}1 {ğ}{{\u{g}}}1 {Ğ}{{\u{G}}}1 {ı}{{\i}}1 {İ}{{\.{I}}}1   {ö}{{\"o}}1 {Ö}{{\"O}}1 {ş}{{\c{s}}}1 {Ş}{{\c{S}}}1 {ü}{{\"u}}1 {Ü}{{\"U}}1
}

\lstset{style=customtex}


\tikzset{%
    terminal/.style={draw, rectangle,
    				 align=center, 
					 minimum height=1cm, 
					 minimum width=2cm,
					 fill=black!10,
					 anchor=mid},
    nonterminal/.style={draw, rectangle,
    					align=left,
					    minimum height=1cm, 
						minimum width=2cm, 
						anchor=mid},% and so on
}

%% Style for terminals
%\tikzstyle{terminal}=[draw, rectangle, 
%					  minimum height=1cm, 
%					  minimum width=2cm, 
%					  fill=black!20,
%					  anchor=south west]
%% Style for nonterminals
%\tikzstyle{nonterminal}=[draw, rectangle, 
%						 minimum height=1 cm, 
%						 minimum width=2 cm, 
%						 anchor=north east]


\newcommand{\HRule}{\rule{\linewidth}{1mm}}
\newcommand{\kutu}[2]{\framebox[#1mm]{\rule[-2mm]{0mm}{#2mm}}}
\newcommand{\gap}{ \\[1mm] }

\newcommand{\Q}{\raisebox{1.7pt}{$\scriptstyle\bigcirc$}}
\newcommand{\minus}{\scalebox{0.35}[1.0]{$-$}}

\setlength{\fboxsep}{10pt}

\tcbsetforeverylayer{enhanced jigsaw, breakable, arc=0mm, boxrule=1pt, boxsep=5pt, after=\vspace{1em}, colback=white, colframe=black}

\newcolumntype{P}[1]{>{\centering\arraybackslash}p{#1}}

\setlength\parindent{0pt}
\newcommand\tab[1][1cm]{\hspace*{#1}}

%\renewcommand\arraystretch{1.2}

\newenvironment{Tab}[1]
  {\def\arraystretch{1}\tabular{#1}}
  {\endtabular}

%%%%%%%%%%%%%%%%%%%%%%%%%%%%%%%%%%%%%%%%%%%%%%%%%%%%%%%%%%%%%%%%%%%%%%%%%%%%%%%%%%%%%%

\title{Discrete Computational Structures \\ Take Home Exam 1}
\author{Ilgaz ŞENYÜZ \\ 2375764} % write your name and id
\date{} % do not write any date

%%%%%%%%%%%%%%%%%%%%%%%%%%%%%%%%%%%%%%%%%%%%%%%%%%%%%%%%%%%%%%%%%%%%%%%%%%%%%%%%%%%%%%

\begin{document}
\HRule\\
Middle East Technical University \hfill Department of Computer Engineering
{\let\newpage\relax\maketitle}
\HRule\\
\vspace{1cm}

%%%%%%%%%%%%%%%%%%%%%%%%%%%%%%%%%%%%%%%%%%%%%%%%%%%%%%%%%%%%%%%%%%%%%%%%%%%%%%%%%%%%%%

% Write your answers below the section tags
\section*{Question 1 \hfill \normalfont{(10 pts)}}

\tab \textbf{a)} Prove that the compound proposition
\begin{equation*}
    \neg(p\wedge q) \leftrightarrow (\neg q \rightarrow p)
\end{equation*} 
    \tab is logically equivalent to
\begin{equation*}
    (p\vee q) \wedge (\neg p \vee \neg q)
\end{equation*} 
\begin{tcolorbox}
1) By the defination of double implication
\begin{displaymath}
(\neg(p\wedge q) \leftrightarrow (\neg q \rightarrow p) \equiv \neg(p \wedge q) \rightarrow ( (\neg q \rightarrow p) ) \wedge ( (\neg q \rightarrow p)\rightarrow \neg(p \wedge q) )
\end{displaymath}
2) By the defination of implies x4
\begin{displaymath}
\equiv ( (p \wedge q) \vee ( q \vee p) ) \wedge ( \neg( q \vee p)\vee \neg(p \wedge q) )
\end{displaymath}
3) By using De Morgan's Law x2
\begin{displaymath}
\equiv ( (p \wedge q) \vee ( q \vee p) ) \wedge ( ( \neg q \wedge \neg p)\vee (\neg p \vee \neg q) )
\end{displaymath}
4) Using Distribution Law x2
\begin{displaymath}
\equiv ( (p \wedge q) \vee q)  \vee (p \wedge q) \vee p) ) \wedge ( ( \neg q \wedge \neg p)\vee \neg p) \vee (( \neg q \wedge \neg p)\vee \neg q) )
\end{displaymath}
5) Using Absorption Law x2
\begin{displaymath}
(q \vee p) \wedge (\neg p \vee \neg q)
\end{displaymath}
Which is equivalent to 
\begin{equation*}
    (p\vee q) \wedge (\neg p \vee \neg q)
\end{equation*} \\
\centering{\textbf{-End of Proof-} }

\end{tcolorbox}

\newpage



\newpage
\section*{Question 2 \hfill \normalfont{(30 pts)}}
Translate the following English sentences into compound predicate logic propositions using the predicates
below. \\ \\
I(x, y): x is an intern in faculty y.\\ 
E(x, y): x has employee id number y.\\
S(x, y): x is supervised by y.\\
A(x, y): x is admitted to job position y.\\
J(x, y): x is a job position in faculty y.\\ \\

\textbf{a.} Two different interns in the same faculty cannot have the same employee id number.\\
\textbf{b.} There are some interns in all faculties who are supervised by no one but themselves.\\
\textbf{c.} At most two interns can be admitted to each job position in the medicine faculty.\\


\begin{tcolorbox}
   \textbf{a)}\item $ \forall x,y,a,b, (x \neq z) [( ( I(x,b) \wedge I(y,b)) \rightarrow (\neg( (E(x,a) \wedge E(z,a) )  ) ]$ \newline
   \textbf{b)}\item $ \forall x, z \exists y, (y\neq z)[ I(y,x) \rightarrow (\neg S(y,z)\wedge S(y,y))]$ \newline
   \textbf{c)}\item $ \forall a,b,c,k\neg \exists x,y,z ([I(x,a)\wedge (A(x,J(k,Medicine)))] \wedge [I(y,b)\wedge (A(x,J(k,Medicine)))] \wedge [I(z,c)\wedge (A(x,J(k,Medicine)))] \wedge x\neq y \wedge x\neq z \wedge y\neq z)$ \newline
\end{tcolorbox}
\newpage
\section*{Question 3a)\hfill \normalfont{}}
\begin{equation*}
p \vee \neg q , p \vee r \vdash (r \rightarrow q) \rightarrow p
\end{equation*}


\begin{center}
\begin{fitchproof}
\fitchline{$p \vee \neg q$}{premise}
\fitchline{$p \vee r$}{premise}
\subproof{
\fitchline{$(r \rightarrow q)$}{assumption}
    \subproof{
    \fitchline{$\neg p$}{assumption}
            \subproof{
                \fitchline{$p$}{assumption}
                \fitchline{$\bot$}{4,5 \neg e}
                \fitchline{$\neg q$}{6 \bot e}
            }
            \subproof{
                \fitchline{$\neg q$}{assumption}
            }
        \fitchline{$\neg q$}{1,5-7,8 \vee e}
            \subproof{
                \fitchline{$p$}{assumption}
                \fitchline{$\bot$}{4,10 \neg e}
                \fitchline{$r$}{11 \bot e}
            }
            \subproof{
                \fitchline{$r$}{assumption}
            }
        \fitchline{$r$}{2,10-12,13 \vee e}
        \fitchline{$q$}{3,14 \rightarrow e}
        \fitchline{$\bot$}{9,15 \neg e}
    }
    \fitchline{$\neg \neg p$}{4-16 \neg i}
    \fitchline{$p$}{17 \neg \neg e}
}
    \fitchline{$(r \rightarrow q) \rightarrow p$}{3-18 \rightarrow i}
\end{fitchproof}
\end{center}
\noindent
\newline

\newpage
\section*{Question 3b) \hfill \normalfont{}}
\begin{equation*}
\vdash ((q \rightarrow p) \rightarrow q) \rightarrow q
\end{equation*}
\begin{center}
\begin{fitchproof}
\subproof{
    \fitchline{$(q \rightarrow p)\rightarrow q$}{assumption}
    \subproof{
        \fitchline{$\neg q$}{assumption}
        \subproof{
            \fitchline{$q$}{assumption}
            \subproof{
                \fitchline{$\neg p$}{assumption}
                \fitchline{$q$}{3 Reiteration(copy)}
                \fitchline{$\neg q$}{2 Reiteration}
                \fitchline{$\bot$}{8 \neg e}
            }
            \fitchline{$\neg \neg p$}{4-7 \neg i}
            \fitchline{$p$}{5,6 \neg \neg e}
        }
        \fitchline{$q \rightarrow p$}{3-9 \rightarrow i}
            \subproof{
                \fitchline{$q \rightarrow p$}{assumption}
                \fitchline{$\neg q$}{2 Reiteration}
            }
        \fitchline{$(q \rightarrow p) \rightarrow \neg q$}{11-12 \rightarrow i}
            \subproof{
                \fitchline{$q \rightarrow p$}{assumption}
                \fitchline{$q$}{1,14 \rightarrow e}
                \fitchline{$\neg q$}{13,14 \rightarrow e}
                \fitchline{$\bot$}{15,16 \neg e}
            }
            \fitchline{$\neg (q \rightarrow p)$}{14-17 \neg i}
            \fitchline{$\bot$}{10,18 \neg e}
    }
    \fitchline{$\neg \neg q$}{3-19 \neg i}
    \fitchline{$q$}{20 \neg \neg e}
}
\fitchline{$((q \rightarrow p)\rightarrow q) \rightarrow q$}{1-21 \rightarrow i}
\end{fitchproof}
\end{center}


\noindent
\newline


\newpage
\section*{Question 4a) \hfill }
\tab 
\begin{equation*}
    \neg \forall x(P(x) \rightarrow Q(x)) \vdash \exists x(P(x) \wedge \neg Q(x))
\end{equation*}
\begin{center}
\begin{fitchproof}
\fitchline{$\neg \forall x(P(x) \rightarrow Q(x))$}{premise}
\subproof{
    \fitchline{$\neg \exists x(P(x) \wedge \neg Q(x))$}{assumption}
    \subproof{
        \fitchline{$fresh name: a$}{}
        \subproof{
            \fitchline{$\neg(P(a) \rightarrow Q(a))$}{assumption}
            \subproof{
                \fitchline{$\neg (P(a) \wedge \neg Q(a))$}{assumption}
                \subproof{
                    \fitchline{$P(a)$}{assumption}
                    \subproof{
                        \fitchline{$\neg Q(a)$}{assumption}
                        \fitchline{$P(a) \wedge \neg Q(a)$}{6,7 \wedge i}
                        \fitchline{$\exists x (P(x) \wedge \neg Q(x))$}{8 \exists i}
                        \fitchline{$\bot$}{2,9 \neg e}
                    }
                    \fitchline{$\neg \neg Q(a)$}{7-10 \neg i}
                    \fitchline{$Q(a)$}{11 \neg \neg e}
                }
                \fitchline{$P(a)\rightarrow Q(a)$}{6-12 \rightarrow i}
                \fitchline{$\bot$}{4,13 \neg e}
            }
            \fitchline{$\neg \neg(P(a) \wedge \neg Q(a))$}{5-14 \neg i}
            \fitchline{$P(a) \wedge \neg Q(a)$}{15 \neg \neg e}
            \fitchline{$\exists x(P(x) \wedge \neg Q(x))$}{16 \exists i}
            \fitchline{$\bot$}{2,17 \neg e}
        }
        \fitchline{$\neg \neg(P(a) \rightarrow Q(a))$}{4-18 \neg i}
        \fitchline{$P(a) \rightarrow Q(a)$}{19 \neg \neg e}
    }
    \fitchline{$\forall x(P(x) \rightarrow Q(x))$}{3-20 \forall i}
    \fitchline{$\bot$}{1,21 \neg e}
}
\fitchline{$\neg \neg \exists x(P(x) \wedge \neg Q(x))$}{2-22 \neg i}
\fitchline{$\exists x(P(x) \wedge \neg Q(x))$}{23 \neg \neg e}
\end{fitchproof}
\end{center}


\noindent
\newline



\end{document}
