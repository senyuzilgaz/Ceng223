\documentclass[a4paper,12pt]{article}

\usepackage{float}


\usepackage[utf8]{inputenc}
\usepackage[dvips]{graphicx}
%\usepackage{a4wide}
\usepackage{epsfig}
\usepackage{fancybox}
\usepackage{verbatim}
\usepackage{array}
\usepackage{latexsym}
\usepackage{alltt}
\usepackage{amssymb}
\usepackage{amsmath,amsthm}
\usepackage{bm}
\usepackage{wasysym}
\documentclass[11pt]{article}
\usepackage[utf8]{inputenc}
\usepackage[dvips]{graphicx}
\usepackage{fancybox}
\usepackage{verbatim}
\usepackage{array}
\usepackage{latexsym}
\usepackage{alltt}
\usepackage{hyperref}
\usepackage{textcomp}
\usepackage{color}
\usepackage{amsmath}
\usepackage{amsfonts}
\usepackage{tikz}
\usepackage{fitch}  % to use fitch
\usepackage{float}
\usepackage[hmargin=3cm,vmargin=5.0cm]{geometry}
%\topmargin=0cm
\topmargin=-2cm
\addtolength{\textheight}{6.5cm}
\addtolength{\textwidth}{2.0cm}
%\setlength{\leftmargin}{-5cm}
\setlength{\oddsidemargin}{0.0cm}
\setlength{\evensidemargin}{0.0cm}

%\usepackage{fullpage}
%\usepackage{hyperref}
\usepackage{listings}
\usepackage{color}
\usepackage{algorithm}
\usepackage{algpseudocode}
\usepackage[hmargin=2cm,vmargin=3.0cm]{geometry}
%\topmargin=0cm
%\topmargin=-1.8cm
%\addtolength{\textheight}{6.5cm}
%\addtolength{\textwidth}{2.0cm}
%\setlength{\leftmargin}{-3cm}
%\setlength{\oddsidemargin}{0.0cm}
%\setlength{\evensidemargin}{0.0cm}

%misc libraries goes here
\usepackage{tikz}
\usepackage{tikz-qtree}
\usetikzlibrary{automata,positioning}

\usepackage{multicol}
\usepackage{enumitem}

\usepackage[most]{tcolorbox}

\usepackage[colorlinks=true,urlcolor=black,linkcolor=black]{hyperref}


\lstdefinestyle{customtex}{
    %backgroundcolor=\color{lbcolor},
    tabsize=2,
    language=TeX,
    numbers=none,
    basicstyle=\footnotesize\ttfamily,
    numberstyle=\footnotesize,
    aboveskip={0.0\baselineskip},
    belowskip={0.0\baselineskip},
    %
    columns=flexible,
    keepspaces=true,
    fontadjust=true,
    upquote=true,
    %
    breaklines=true,
    prebreak=\raisebox{0ex}[0ex][0ex]{\ensuremath{\hookleftarrow}},
    frame=single,
    showtabs=false,
    showspaces=false,
    showstringspaces=false,
    %
    %identifierstyle=\color[rgb]{0,0.2,0.8},
    identifierstyle=\color[rgb]{0,0,0.5},
    %identifierstyle=\color[rgb]{0.133,0.545,0.133},
    %keywordstyle=\color[rgb]{0.8,0,0},
    %keywordstyle=\color[rgb]{0.133,0.545,0.133},
    keywordstyle=\color[rgb]{0,0,0.5},
    %commentstyle=\color[rgb]{0.133,0.545,0.133},
    commentstyle=\color[rgb]{0.545,0.545,0.545},
    %stringstyle=\color[rgb]{0.827,0.627,0.133},
    stringstyle=\color[rgb]{0.133,0.545,0.133},
    %
    literate={â}{{\^{a}}}1 {Â}{{\^{A}}}1 {ç}{{\c{c}}}1 {Ç}{{\c{C}}}1 {ğ}{{\u{g}}}1 {Ğ}{{\u{G}}}1 {ı}{{\i}}1 {İ}{{\.{I}}}1   {ö}{{\"o}}1 {Ö}{{\"O}}1 {ş}{{\c{s}}}1 {Ş}{{\c{S}}}1 {ü}{{\"u}}1 {Ü}{{\"U}}1 {~}{$\sim$}{1}
}

\lstdefinestyle{output}{
    %backgroundcolor=\color{lbcolor},
    tabsize=2,
    numbers=none,
    basicstyle=\footnotesize\ttfamily,
    numberstyle=\footnotesize,
    aboveskip={0.0\baselineskip},
    belowskip={0.0\baselineskip},
    %
    columns=flexible,
    keepspaces=true,
    fontadjust=true,
    upquote=true,
    %
    breaklines=true,
    prebreak=\raisebox{0ex}[0ex][0ex]{\ensuremath{\hookleftarrow}},
    frame=single,
    showtabs=false,
    showspaces=false,
    showstringspaces=false,
    %
    %identifierstyle=\color[rgb]{0.44,0.12,0.1},
    identifierstyle=\color[rgb]{0,0,0},
    keywordstyle=\color[rgb]{0,0,0},
    commentstyle=\color[rgb]{0,0,0},
    stringstyle=\color[rgb]{0,0,0},
    %
    literate={â}{{\^{a}}}1 {Â}{{\^{A}}}1 {ç}{{\c{c}}}1 {Ç}{{\c{C}}}1 {ğ}{{\u{g}}}1 {Ğ}{{\u{G}}}1 {ı}{{\i}}1 {İ}{{\.{I}}}1   {ö}{{\"o}}1 {Ö}{{\"O}}1 {ş}{{\c{s}}}1 {Ş}{{\c{S}}}1 {ü}{{\"u}}1 {Ü}{{\"U}}1
}

\lstset{style=customtex}


\tikzset{%
    terminal/.style={draw, rectangle,
    				 align=center, 
					 minimum height=1cm, 
					 minimum width=2cm,
					 fill=black!10,
					 anchor=mid},
    nonterminal/.style={draw, rectangle,
    					align=left,
					    minimum height=1cm, 
						minimum width=2cm, 
						anchor=mid},% and so on
}

%% Style for terminals
%\tikzstyle{terminal}=[draw, rectangle, 
%					  minimum height=1cm, 
%					  minimum width=2cm, 
%					  fill=black!20,
%					  anchor=south west]
%% Style for nonterminals
%\tikzstyle{nonterminal}=[draw, rectangle, 
%						 minimum height=1 cm, 
%						 minimum width=2 cm, 
%						 anchor=north east]


\newcommand{\HRule}{\rule{\linewidth}{1mm}}
\newcommand{\kutu}[2]{\framebox[#1mm]{\rule[-2mm]{0mm}{#2mm}}}
\newcommand{\gap}{ \\[1mm] }

\newcommand{\Q}{\raisebox{1.7pt}{$\scriptstyle\bigcirc$}}
\newcommand{\minus}{\scalebox{0.35}[1.0]{$-$}}

\setlength{\fboxsep}{10pt}

\tcbsetforeverylayer{enhanced jigsaw, breakable, arc=0mm, boxrule=1pt, boxsep=5pt, after=\vspace{1em}, colback=white, colframe=black}

\newcolumntype{P}[1]{>{\centering\arraybackslash}p{#1}}

\setlength\parindent{0pt}
\newcommand\tab[1][1cm]{\hspace*{#1}}

%\renewcommand\arraystretch{1.2}

\newenvironment{Tab}[1]
  {\def\arraystretch{1}\tabular{#1}}
  {\endtabular}

%%%%%%%%%%%%%%%%%%%%%%%%%%%%%%%%%%%%%%%%%%%%%%%%%%%%%%%%%%%%%%%%%%%%%%%%%%%%%%%%%%%%%%

\title{Discrete Computational Structures \\ Take Home Exam 1}
\author{Ilgaz ŞENYÜZ \\ 2375764} % write your name and id
\date{} % do not write any date

%%%%%%%%%%%%%%%%%%%%%%%%%%%%%%%%%%%%%%%%%%%%%%%%%%%%%%%%%%%%%%%%%%%%%%%%%%%%%%%%%%%%%%

\begin{document}
\HRule\\
Middle East Technical University \hfill Department of Computer Engineering
{\let\newpage\relax\maketitle}
\HRule\\
\vspace{1cm}

%%%%%%%%%%%%%%%%%%%%%%%%%%%%%%%%%%%%%%%%%%%%%%%%%%%%%%%%%%%%%%%%%%%%%%%%%%%%%%%%%%%%%%

% Write your answers below the section tags
\section*{Question 1 \hfill \normalfont{(25 pts)}}

\tab \textbf{a)} Given the sets A and B, prove that
\begin{equation*}
(A \cup B) - (A \cap B) = (A - B) \cup (B - A)
\end{equation*} 
\tab using set membership notation and logical equivalences. Show each step clearly. \\ \\ \\
We must show both \begin{equation*}
(A \cup B) - (A \cap B) \subseteq (A - B) \cup (B - A)
\end{equation*} 
and \begin{equation*}
 (A - B) \cup (B - A) \subseteq 
(A \cup B) - (A \cap B) \end{equation*} 

\begin{tcolorbox}
1)
Suppose  x $\in  (A \cup B) - (A \cap B)$ \\

2) By the definition of difference
\begin{displaymath}
 x \in (A \vee B) \wedge x \not\in (A \cap B)
\end{displaymath}

3) By the defination of union and intersection
\begin{displaymath}
 (x \in A \vee  x \in B) \wedge \neg(x \in A \wedge x\in B)
\end{displaymath}
4) By using De Morgan's Law for propositions
\begin{displaymath}
(x \in A \vee  x \in B) \wedge (\neg (x \in A) \wedge \neg(x\in B))
\end{displaymath}
5) By the definition of $\not\in$
\begin{displaymath}
(x \in A \vee  x \in B) \wedge ( x \not\in A \vee x \not\in B)
\end{displaymath}
6) Using Distributive Law 
\begin{displaymath}
((x \in A \vee  x \in B) \wedge x\not\in A) \vee ((x \in A \vee  x \in B) \wedge x\not\in B)
\end{displaymath}
7) Using Distributive Law x2
\begin{displaymath}
[(x \in A \wedge x \not\in A) \vee (x \in B \wedge x \not\in A)] \vee [(x \in A \wedge x \not\in B) \vee (x \in B \wedge x \not\in B)]
\end{displaymath}
8) Using Complement Laws
\begin{displaymath}
[\varnothing \vee (x \in B \wedge x \not\in A)] \vee [(x \in A \wedge x \not\in B) \vee \varnothing]
\end{displaymath}
9) Using Identity Law
\begin{displaymath}
 (x \in B \wedge x \not\in A) \vee (x \in A \wedge x \not\in B)
\end{displaymath}
10) By the definition of difference x2
\begin{displaymath}
(1)x\in(B-A) \vee (2)x \in (A-B)
\end{displaymath}
11) Using Commutative Law
\begin{displaymath}
(2)x \in (A-B) \vee  (1)x\in(B-A)
\end{displaymath}
12) By the definition of union
\begin{displaymath}
(A - B) \cup (B - A)
\end{displaymath}
Therefore $(A \cup B) - (A \cap B) \subseteq (A - B) \cup (B - A)$
\end{tcolorbox}

\begin{tcolorbox}
1)
Suppose  x $\in  (A - B) \cup (B - A)$ \\

2) By the definition of union
\begin{displaymath}
x\in(A-B) \vee x \in (B-A)
\end{displaymath}

3) By the definition of difference x2
\begin{displaymath}
 (x \in A \wedge x \not\in B) \vee  (x \in B \wedge x \not\in A) 
\end{displaymath}
4) Using Distribution Law 
\begin{displaymath}
((x\in A \wedge x\not\in B) \vee x \in B) \wedge ((x\in A \wedge x\not\in B) \vee x \not\in A) 
\end{displaymath}
5) Using Distribution Law x2
\begin{displaymath}
[(x\in A \vee x \in B) \wedge (x\not\in B \vee x\in B)] \wedge
[(x\in A \vee x \not\in A) \wedge (x\not\in B \vee x\not\in A)]
\end{displaymath}
6) Using Complement Law x2
\begin{displaymath}
[(x\in A \vee x \in B) \wedge U] \wedge
[U \wedge (x\not\in B \vee x\not\in A)]
\end{displaymath}
7) Using Identity Law x2
\begin{displaymath}
(x\in A \vee x \in B) \wedge
(x\not\in B \vee x\not\in A)
\end{displaymath}
8) By the definition of $\not\in$
\begin{displaymath}
(x \in A \vee  x \in B) \wedge (\neg (x \in B) \vee \neg(x\in A))
\end{displaymath}
9) By the definition of De Morgan's Law for propositional logic
\begin{displaymath}
(x \in A \vee  x \in B) \wedge \neg (x \in B \wedge x\in A)
\end{displaymath}
10) By the definition of Union and Intersection
\begin{displaymath}
 x \in (A \cup B) \wedge x \not\in (B \cap A)
\end{displaymath}
11) Using Commutative Law
\begin{displaymath}
 x \in (A \cup B) \wedge x \not\in (A \cap B)
\end{displaymath}
12) By the definition of difference
\begin{displaymath}
 (A \cup B)-(A \cap B)
\end{displaymath}
Therefore $(A - B) \cup (B - A) \subseteq (A \cup B) - (A \cap B)$
\end{tcolorbox}

Hence, $
(A \cup B) - (A \cap B) = (A - B) \cup (B - A)
$

\newpage



\newpage
\section*{Question 2 \hfill \normalfont{(25 pts)}}
Prove that the set

\{f $|$ f : N $\rightarrow$ \{0, 1\}, f is a function\} - \{f $|$ f : \{0, 1\} $\rightarrow$ N, f is a function\}
is uncountable. \\\\
Let say X to the set : \{f $|$ f : N $\rightarrow$ \{0, 1\}, f is a function\} \\\\
Let say Y to the set : \{f $|$ f : \{0, 1\} $\rightarrow$ N, f is a function\}

\begin{tcolorbox}
Any function f, where N $\rightarrow$ \{0, 1\} means the set of all binary strings. \\ I will show that this set is uncountable by the diagonalization argument.\\ The set X \{includes a1, a2, a3 ...\} where: \\
a1= a.11 a.12 a.13 ... \\
a2= {a.21 a.22 a.23 ... \\
a3= a.31 a.32 a.33 ... \\
... goes on like this (any enumeration)\\
And there exists a binary string b = b1 b2 b3 ... such that: \\
b.i $\neq$ a.ii, For instance b.i = 0 if a.ii = 1, and b.i = 1 if a.ii = 0 \\
b $\neq $ ai, i $\in N$ (a1, a2, a3 ...) \\ \\
Therefore there does not exist an enumeration counting each
element in X \\\\
Hence the set X is UNCOUNTABLE.

\end{tcolorbox}


\begin{tcolorbox}
Any function f, where \{f $|$ f : \{0, 1\} $\rightarrow$ N, is determined by its values at 0 and 1, f(0) and f(1). f(0) can be any Natural number so does f(1). Therefore f is actually cartesian product of two sets with cardinality of N. Lets enumarate these sets as \{a1,a2,a3...\} and \{b1,b2,b3...\}. I arrange these elements in an infinite matrix and use "zigzag" method to traverse this matrix. For instance $\{(a1,b1), (a2,b1), (a1,b2), (a3,b1), (a2,b2), (a1,b3)...\}$. Hence f is countable. \\\\ Therefore the set Y is COUNTABLE.

\end{tcolorbox}

\begin{tcolorbox}
By the previous two boxes, X is UNCOUNTABLE, and Y is COUNTABLE.\\\\
Assume X - Y is countable.\\
The union of countably many countable sets is countable; thus (X-Y)$\cup$Y is countable. (1)\\
Since X \subset (X-Y)$\cup$Y , then X must be countable too.
$\bot$But it can not be, becuase we know that X is uncountable.\\\\

Therefore  X - Y is UNCOUNTABLE\\

(Proof of (1): \\
Since (X-Y) is countable, we can enumerate (X-Y)=\{a1,a2,a3,...\}.\\
Since Y is countable we can enumerate Y=\{b1,b2,...\}\\
And now we can enumerate (X-Y)$\cup$Y as \{a1,b1,a2,b2,...\} and thus 
(X-Y)$\cup$Y is countable.) \\ \\



\end{tcolorbox}

In the previous box, it is shown that X-Y which is  \{f $|$ f : N $\rightarrow$ \{0, 1\}, f is a function\} - \{f $|$ f : \{0, 1\} $\rightarrow$ N, f is a function\} is UNCOUNTABLE

\newpage
\section*{Question 3)\hfill \normalfont{25pts}}
Prove that $f(n) = 4^n +5n^2 logn $ is not $O(2^n)$

\begin{center}
\begin{tcolorbox}
    Definition: f(x) is O(g(x)) if there exists such constants c and k such that f(x) $f(x) \le c \cdot g(x)$ where $x \ge k $  \\ \\
    Assume $f(n) = 4^n +5n^2 logn $ is  $O(2^n)$, and $n>1$\\
    $\frac{f(n)}{g(n)} \le c,$ where $\frac{f(n)}{g(n)}$ = $\frac{2^{2n}+5n^2logn}{2^n} = 2^n + \frac{5n^2logn}{2^n}$\\ \\
    Since $x > 1$, \tab $2^n + \frac{5n^2logn}{2^n} > 2^n +0 = 2^n$\\ \\
    $n > \log_2(c)$ implies $2^n > c$ and $f(n) > c \cdot 2^n$ \\ \\
    $\bot$ Contradiction with the definition of the Big O Notation  \\ \\
    Therefore when $n > 1, n > k$, and $n > \log_2(c)$ f(n)$> c\cdot 2^n$ which contradicts with the definition of the Big O Notation. \\\\
    $f(n) = 4^n +5n^2 logn $ is NOT $O(2^n)$
\end{tcolorbox}
\end{center}
\noindent
\newline

\newpage

\newpage
\section*{Question 4) \hfill \normalfont{25pts}}
\tab 

\begin{center}
\begin{equation*}
    x>2, n>2, (2x-1)^n-x^2 \equiv -x-1 (mod(x-1))
\end{equation*}
Determine the value of x
\begin{tcolorbox}
$
    1 \tab 2x-1 = 1+(2\cdot(x-1)) \hfill \normalfont{}\\ \\
    2 \tab 2x-1\equiv 1 (mod(x-1)) \hfill \normalfont{( 1)}\\ \\
    3  \tab x\equiv 1 (mod(x-1))\hfill \normalfont{( 2)}\\ \\
    4 \tab   (2x-1)^n-x^2 \equiv 1^n - 1^2 \hfill \normalfont{(2,3)}\\ \\
    5  \tab 0 \equiv -x -1 (mod(x-1)) \hfill \normalfont{(Premise,4)}\\ \\
    6  \tab x+1 \equiv 0 (mod(x-1)) \hfill \normalfont{(5)}\\ \\
   7 \tab 2 \equiv 0 (mod(x-1)) \hfill \normalfont{(3,6)} \\ \\
   8 \tab 2 \equiv x-1 (mod(x-1)) \hfill \normalfont{(7)} \\ \\
      9 \tab x =  3  \hfill \normalfont{(x>2,8)} \\ \\
$
\end{tcolorbox}
\end{center}


\noindent
\newline



\end{document}
